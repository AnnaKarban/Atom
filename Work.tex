\documentclass[a4paper]{article}
\usepackage[14pt]{extsizes}
\usepackage[utf8]{inputenc}
\usepackage[ukrainian]{babel}
\usepackage{setspace,amsmath}
\usepackage{multirow}
\usepackage{makecell}
\usepackage{comment}
\usepackage{ltxtable}

\usepackage[unicode, pdftex]{hyperref}
\hypersetup{
    colorlinks=true,
    linkcolor=blue}
\usepackage[left=20mm, top=15mm, right=15mm, bottom=15mm, nohead, footskip=10mm]{geometry}

\begin{document}
\thispagestyle{empty}
\begin{center}
\hfill \break
\normalsize{Міністерство освіти і науки України}\\
\normalsize{Ліцей інформаційних технологій}\\
\normalsize{Кафедра з інформатики}\\
\hfill\break
\hfill \break
\hfill \break
\hfill \break
\large{\textbf{Курсова робота}}\\
\normalsize{\textbf{З дисципліни:}}\\
\textbf{На тему: "Дослідження різних підходів до підготовки формул у вигляді "плоского" тексту за допомогою спеціалізованих програмних засобів."}\\
\end{center}

\vspace{\stretch{1}}

\begin{flushright}
\normalsize{Виконала:}\\
\normalsize{учениця 10-В класу}\\
\normalsize{Карбан Анна Романівна}\\
\normalsize{Перевірив:}\\
\normalsize{Хижа Олександр Леонідович}\\
\end{flushright}


\begin{center}
\vspace{\stretch{1}}
  Дніпро-2022 \pagebreak
\end{center}
 % выключаем отображение номера для этой страницы

\newpage
\begin{center}
   \tableofcontents % Вывод содержания
   \label{sec:}
   \ref{sec:}
\end{center}

\newpage
\section{Анотації}
\newpage
\section{Вступ}
Детальніше тут\cite{LateX_Baldin}
\newpage
\section{Постановка задачі}
\begin{center}
\begin{longtable}{|c|}
\hline
\multicolumn{1}{|c|}{\textbf{Заголовок}}  \\
\hline
\endfirsthead

%\multicolumn{1}{c}%
%{{ \tablename\ \thetable{} }} \\
\hline
\multicolumn{1}{|c|}{\textbf{Заголовок}}  \\
\hline
\endhead
\hline
\multicolumn{1}{|r|}{{Continued on next page}} \\ \hline
\endfoot
\hline
\endlastfoot

\makecell[{{p{3cm}}}]{111}\\
\makecell[l]{1}\\
1\\
1\\
1\\
1\\
1\\
1\\
1\\
1\\
1\\
1\\
1\\
1\\
1\\
1\\
1\\
1\\
1\\
1\\
1\\
1\\
1\\
1\\
1\\
1\\
1\\
1\\
1\\
1\\
1\\
1\\
1\\
1\\
1\\
1\\
1\\
1\\
1\\
1\\
1\\
1\\
1\\
1\\
1\\
\end{longtable}
\end{center}

\newpage
\section{Основна частина}
\begin{equation}
  \label{eq1}
   x^2+2x-5=0\\
\end{equation}
Подивимось на \ref{eq1}\\
^5/_7\\
\sqrt{\frac{5}{10}}\\
\newpage

\section{Висновок}
%\setlength{\tabcolsep}{10pt}%расстояние между столбцами
\begin{center}
\begin{tabularx}{300pt}{|c|X|c|X|c|X|}
\hline
\multicolumn{3}{|c|}{\textbf{Заголовок}} \\
\hline
%\rule{0cm}{2cm} or \\[1cm] изменение высоты
& Четные5565\-вапвапап & 2, 4, 6, 8 \\
\cline{2-3}
\raisebox{1.5ex}[0cm][0cm]{Цифры}
& \makecell[{{p{\hsize}}}]{Нечетные чиbbbbbkjkjсла} & 1, 3, 5, 7, 9 \\
\hline
\end{tabularx}
\end{center}
\newpage

\section{Література}
\begin{thebibliography}{9}
\bibitem{LateX_Baldin}
Балдин Е.М. Компьютерная типография LaTeX. - Новосибирск : \\Идательство БХВ-Петербург. - 2008. -308с.
\end{thebibliography}
\newpage

\section{Додатки}
нечетные нечетные нечетные нечетные нечетные нечетные нечетные  нечетные нечетные нечетные нечетные нечетные нечетные нечетные нечетные нечетные нечетные нечетные нечетные нечетные нечетные нечетные
\end{document}
